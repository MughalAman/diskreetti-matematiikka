\documentclass{report}

\input{../preamble}
\input{../macros}
\input{../letterfonts}

\title{\Huge{Diskreetti matematiikka} Tehtävät 2}
\author{\huge{Aman Mughal}}
\date{20/01/2023}

\begin{document}

\maketitle

\qs{6. Formalisoi edellisen kerran tehtävän nro 3 lauseet}{
    \begin{enumerate}[label=\alph*.] 
        \item Ulkona sataa ja tuulee.
        \item Tarja on erehtynyt tai tansseja ei ole tänään.
        \item Jos Merja on oikeassa, niin Juha soittaa kitaraa.
        \item Jos kissa kehrää, niin se on tyytyväinen.
    \end{enumerate}
    Vinkki: Lauseen formalisointi \\
- yhdistä atomilauseet oikeilla konnektiiveilla. \\
- kohta b: usein atomilauseet esitetään aina positiivisina versioina (eli vältetään ’ei’ sanoja )
}

\pf{Vastaukset}{
    \begin{align*}
        \text{a) } & \text{(Sade) = 1 ja (Tuuli) = 1}, \text{ Sade} \land \text{Tuuli} \\
        \text{b) } & \text{(Tarja erehtyy) = 1 tai (Ei ole tansseja) = 1}, \text{ Tarja erehtyy} \lor \text{Ei ole tansseja} \\
        \text{c) } & \text{(Merja oikeassa) = 1 niin (Juha soittaa kitaraa) = 1}, \text{ Merja oikeassa} \rightarrow \text{Juha soittaa kitaraa} \\
        \text{d) } & \text{(Kissa kehrää) = 1 niin (Kissa on tyytyväinen) = 1}, \text{ Kissa kehrää} \rightarrow \text{Kissa on tyytyväinen} \\
    \end{align*}
}

\qs{7. Sulkujen käytöstä hieman}{
    Olkoon lause "Matti Näsä ei ole nuori Toyotan omistaja". \\
    Se formalisoituu muotoon ¬(N $\land$ T). [N: Matti on nuori, T: Matilla on Toyota Mark II] Vastaa seuraaviin kysymyksiin:
    \begin{enumerate}[label=\alph*.]
        \item Voiko nuori Matti omistaa Toyotan? (yo formalisoinnin perusteella, eiköhän tosielämän Matti Näsä ole henkeen ja vereen Toyota-mies)
        \item Jos sulut otetaan pois, mitä seuraavat formalisoinnit vastaavat: $\neg N \land T ja \neg N \land \neg T$ ?
    \end{enumerate}
}

\pf{Vastaukset}{
    \begin{align*}
        \text{a) } & \text{Ei, koska Matti ei ole nuori ja hänellä ei ole Toyotaa.} \\
        \text{b) } & \neg N \land T = \text{Matti ei ole nuori ja hänellä on Toyota.}\\
        & \neg N \land \neg T = \text{Matti ei ole nuori ja hänellä ei ole Toyotaa.}
    \end{align*}
}

\qs{8. Formalisoi lauselogiikan avulla seuraavat luonnolliset lauseet}{
    \begin{enumerate}[label=\alph*.]
        \item Jos nukun hyvin, niin olen aamulla virkeä.
        \item Joko nukun tai olen aamulla väsynyt.
        \item Jos elämässä on säpinää ja rahaa riittää, niin olen iloinen.
    \end{enumerate}
}

\pf{Vastaukset}{
    \begin{align*}
        \text{a) } & \text{(Nukun hyvin)} \rightarrow \text{(Olen aamulla virkeä)} \\
        \text{b) } & \text{(Nukun hyvin)} \lor \text{(Olen aamulla väsynyt)} \\
        \text{c) } & (\text{(Säpinää elämässä)} \land \text{(Rahaa riittää)}) \rightarrow \text{P(Olen iloinen)} \\
    \end{align*}
}

\newpage

\qs{9. Formalisoi seuraavat lauseet (eivät ole yksinkertaisia)}{
    \begin{enumerate}[label=\alph*.]
        \item Tuulee, jos sataa.
        \item Sataa vain, jos tuulee.
    \end{enumerate}
}

\pf{Vastaukset}{
    \begin{align*}
        \text{a) } & \text{(Tuuli)} \leftarrow \text{(Sade)} \\
        \text{b) } & \text{($\neg$Tuuli)} \lor \text{(Sade)} \\
    \end{align*}
}

\qs{10. Totuusarvotaulukko voidaan muodostaa myös useammalle kuin 2 atomilauseelle}{
    Millä alkuarvoilla seuraava lauseke on tosi: $\neg ( A \lor B ) \land \neg C$
}

\pf{Vastaus}{
    %Create a table with 5 columns and 2 rows
    \begin{tabular}{|c|c|c|c|c|}
        \hline
        A & B & C & $\neg ( A \lor B )$  & $\neg ( A \lor B ) \land \neg C$ \\
        \hline
        \hline
        0 & 0 & 0 & 1 & 1 \\
        0 & 0 & 1 & 1 & 1 \\
        \hline
    \end{tabular}
    \newline
    \newline
    \newline
}

\qs{ 11. Ystävykset ja taskulamppu}{
    Neljä ystävystä tuli pimeän tunnelin suulle ja heillä oli vain 1 taskulamppu. Ilman taskulamppua tunnelissa ei pystynyt kulkemaan. Taskulamppu ilmoitti, että sillä riitti virtaa vielä 30 minuuttia. Ystävykset uskalsivat kulkea tunnelin läpi erilaisella vauhdilla. Uljas kulki tunnelin läpi (yhteen suuntaan) 3 minuutissa. Annelta meni vastaavasti aikaa 5, Pekalta 9 ja Vienolta 11 minuuttia. Turvallisuussyistä tunneliin pääsi korkeintaan 2 henkilöä kerrallaan. Tunnelissa kuljettiin tietenkin hitaamman henkilön vauhdin mukaan. Kuinka kaikki ystävykset selviävät ajoissa toiselle puolelle?
}
\noindent
Vastaus:
\begin{enumerate}
    \item erä (taskulampussa on 30 min virtaa) \\ Uljas ja Anne (5 min) $\Leftrightarrow$ Uljas (3 min) \\ Anne on toisella puolella
    \item erä (taskulampussa on 22 min virtaa) \\ Pekka ja Vieno (11 min) $	\Leftrightarrow$ Anne (5 min) \\ Pekka ja Vieno on toisella puolella
    \item erä (taskulampussa on 6 min virtaa) \\ Uljas ja Anne (5 min) $\rightarrow$ poistuminen \\
\end{enumerate}

Virtaa jäljellä: 1 min


\end{document}
